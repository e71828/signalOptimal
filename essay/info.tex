\PaperTitle{基于KCCA算法的EEG和语音特征融合情绪识别} % Article title


\Authors{刘晴\textsuperscript{1}*} % Authors

\affiliation{\quad\textsuperscript{1}\textit{西北工业大学}\qquad*\textbf{通讯作者}: elbox@qq.com}

\Abstract{\phantom{田田}
情绪识别是智能人机交互的重要环节。传统方法主要采用单模态情感特征进行识别,情感识别率低。针对该问题,本文提出了一种核典型相关分析算法(KCCA)的多特征(multi-features)融合情感识别方法。在特征层面上,分别选取外在直观表达信号——语音信号和生理信号——脑电波进行特征提取,然后利用两种特征互补性,采用核典型相关分析算法(KCCA)将它们进行融合,降低特征向量的维数。 最后选择SVM模型对情感识别的训练集进行建模,并通过具体情感数据集进行仿真实验.实验结果表明,核相关分析算法有效的提高了情感识别的正确率。
}


\Keywords{\phantom{田田} 情感识别\quad 核典型相关分析算法\quad 特征融合\quad 脑电特征}

